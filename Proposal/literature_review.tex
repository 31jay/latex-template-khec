\chapter{Literature Review}

In this era of advanced technologies, the world is caught up in a single computer or digital device. Every moment and all human activities are associated with the digital platform. Aside, literature is the magic that can overcome sorrows, pain, and loneliness and motivates in every failure and further encourages achieving the goal. In the digital era, the development of applications provides an amazing platform for organizing and creating literature.  

Margaret Atwood states, "As we embrace technology, we must ensure that literature remains at the heart of the digital age. Software development should enhance storytelling rather than overshadow it, preserving the beauty of language and the power of imagination."

Applications like memo in early keypad phones have evolved into platforms such as Google Keep, Evernote, Microsoft OneNote, Todoist, and others, which highlight the importance of the habit of note-taking. On the other hand, Evernote, Instapaper, Pocket, Flipboard, Readwise, etc., are popular applications for accessing and reading literature content.  

"Notes are stored in virtual 'notebooks' and can be tagged, annotated, edited, searched, and exported" \cite{evernote}.  
"You can create, edit, and share notes with Google Keep" \cite{googlekeep}.  

Neil Gaiman states, "Software development is an ally to authors, empowering us to explore innovative narrative structures and collaborate with readers in unprecedented ways. The digital realm offers endless possibilities to enhance and expand the reader's experience."

Salman Rushdie adds, "Technology has reshaped the literary landscape, enabling us to transcend traditional boundaries. Through software development, we can create multi-dimensional narratives and amplify diverse voices, fostering a truly global literary community."

From the above statements from prominent authors, as a user, every individual wants a well-organized application that not only allows one to create notes and general articles but also provides a wide range of features that enhance the overall literature experience. Google Keep focuses on keeping, editing, and sharing notes, while Evernote facilitates reading articles more extensively.  

However, existing applications lack features for creating typical literature projects, classifying literature by genre, setting nicknames, tracking analytics, adding collaborators securely, managing tags, and searching for prominent literature figures on the same platform.  

To address these issues and promote the journey through the magical world of literature, the project \textbf{LEAF} has been developed.  


\noindent\textbf{References}

\begin{enumerate}
    \item \href{https://en.wikipedia.org/wiki/Evernote}{Evernote. Wikipedia.}
    \item \href{https://support.google.com/keep/answer/2888240?hl=en&co=GENIE.Platform%3DAndroid}{Google Keep Support.}
    \item BE Computer Proposal 3rd Sem [Ankit Rawal (770303)]
\end{enumerate}
